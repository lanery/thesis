% \section{Introduction}
% \label{sec:2.1_intro}

% A central objective within neuroscience and cell biology is to produce high-resolution (\SIrange{1}{10}{\nano\meter}), three-dimensional reconstructions of biological specimen. Volume electron microscopy (EM) is the preferred imaging method in this arena because of its unique ability to resolve features across a wide spectrum of spatial scales \cite{peddie2014exploring, kornfeld2018progress}. While EM provides highly relevant structural information and precise localization of targets, immunogold labeling can only be visualized at high resolution, and in Tokuyasu labeling section areas are typically limited to \SI{0.01}{\milli\meter^2} for analysis \cite{liou1996improving, van2008correlative}. Fluorescence microscopy (FM) provides biologically relevant information by tagging specific biomolecules with fluorescent labels at large scale \cite{giepmans2006fluorescent}. Regions of interest (ROI) can in this way be quickly and reliably identified for subsequent high magnification EM imaging. The information from these two imaging modalities are combined in correlative light and electron microscopy (CLEM). ROI retrieval across different microscopes is, however, nontrivial at large scales, particularly when spread across multiple sections \cite{polishchuk2000correlative, bishop2011near, karreman2014correlating, collinson2017correlating, booth2019superclem}. Other challenges associated with CLEM include the reliance on fiducial markers and intermediate sample preparation \cite{de2015correlated, karreman2016intravital}. An alternative means of combating these challenges is by merging these separate imaging systems into a single, integrated fluorescence and electron microscope \cite{liv2013simultaneous}. By detecting fluorescence expression \textit{in-situ}, it can further be decided in an automated fashion which areas to scan at high magnification and which areas to omit for the sake of higher throughput \cite{delpiano2018automated}. For array tomography applications, ROIs can be targeted with increasing magnification through a sequence of feedback loops \cite{gabarre2021workflow}. Similarly, strategies for rapidly screening sections have been developed for sequential CLEM to limit volume acquisitions to select ROI \cite{burel2018targeted}.

% Despite the potential benefits, an integrated microscope presents new challenges. In conventional array tomography sample preparation, the sample is eluted and restained between imaging methods \cite{micheva2007array}. Hence, there is no need to preserve fluorescence labelling, which allows for post-staining to enhance EM contrast \cite{watson1958staining,tapia2012high}. The traditional way to compensate for diminished contrast is to boost the EM signal by increasing the dwell time per pixel, but this comes at the expense of throughput. An additional complication in integrated CLEM is electron-beam-induced quenching of the fluorescence \cite{srinivasa2021electron}. This imposes the constraint that the fluorescence in a given area must be acquired prior to exposure from the electron beam, which prohibits uniformly pre-irradiating the sample with the electron beam to enhance and stabilize contrast \cite{kuipers2015scanning}. Conversely, in conventional serial-section EM, there are scarce constraints regarding the number of times a particular sample may be scanned, making possible approaches such as that by \textcite{hildebrand2017whole}.

% Our goal is to establish a workflow capable of quickly and efficiently rendering three-dimensional CLEM volumes from serial sections in such a way as to overcome these challenges. Three key initiatives steered the design of our integrated correlative array tomography (iCAT) procedure. First, to prevent damaging or quenching of the fluorescence signal via electron-beam irradiation, each FM field of view must be acquired prior to EM exposure. Second, to compensate for the reduced application of contrast agents, backscattered electron (BSE) collection efficiency is enhanced via a negative stage bias, allowing for higher throughput \cite{bouwer2016deceleration,lane2021optimization}. Finally, a high precision EM-FM overlay is facilitated by the use of cathodoluminescent (CL) points, which eliminates the need for artificial fiducial markers \cite{haring2017automated}. An alignment method was then developed to reconstruct the correlative image stack. Islets of Langerhans from both rat and zebrafish pancreas tissue were chosen to prototype the imaging and reconstruction workflows. By offering a more holistic visualization of tissue, our integrated approach to 3D CLEM could lead to greater insights in (patho)biology \cite{de2020large}.
