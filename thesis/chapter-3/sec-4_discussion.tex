\section{Discussion}
\label{sec:3.4_discussion}

A new workflow for integrated array tomography for the semi-automated acquisition and reconstruction of volume CLEM data is presented. High-resolution EM is limited to select ROI by targeting areas based on fluorescence expression. This not only expedites acquisition time, but eases the burden on data management requirements. Interpretation of EM data is in turn facilitated by the addition of fluorescent labels. The workflow demonstrated here extends the work of \textcite{liv2013simultaneous}, which introduced the integrated microscope, and \textcite{haring2017automated}, which presented the fiducial-free CL registration procedure, to targeted correlative imaging of serial sections. \textcite{gabarre2021workflow} presented an alternative method for integrated array tomography in which light microscopy and EM are combined to localize structures through a series of feedback loops. Our approach differs in several ways. First, fluorescence imaging is done \textit{in-vacuo} as opposed to transmitted light microscopy done at ambient pressure. This allows for more automated EM-FM (or EM-LM) overlay, as the CL registration procedure can only be done in high vacuum \cite{haring2017automated}. Additionally, the multi-modal alignment methodology conceived here offers a more scalable solution for generating volumetric CLEM data. Integrated array tomography was inspired in part by the multi-scale approach of \textcite{hildebrand2017whole}, in which full brain EM imaging of a larval zebrafish was conducted by selecting ROI for subsequent acquisition based on inspection between imaging rounds. In this work, conversely, ROI are identified by \textit{in-situ} fluorescence, bypassing the need for post-processing and alignment between magnification scales.

On-section immunofluorescence and fluorescent staining constitute viable options for FM imaging of resin-embedded sections in high vacuum. Pancreas tissue in particular is well-suited for immunofluorescence due to the prevalence of insulin epitopes. While in both nature and technique development, immunolabeling approaches are always dependent on the capacity for antibodies and epitopes to interact, this is typically inefficient for most antibodies, and particularly so for EPON-embedded sections. We find that approximately 1 in 10 antibodies tested in our lab are applicable for EPON labeling. While acrylic resins (e.g. Lowicryl, LR White) have been shown to be more compatible with immunolabeling, a trade-off must be made between the strength of the fluorescence signal and the quality of the ultrastructure \cite{watanabe2011protein, paez2015fixation}. Complications with serial sectioning and ultrastructure preservation (beyond that shown in the zebrafish pancreas) arose when experimenting with Lowicryl; hence EPON was selected as the embedding medium for this study.

Probes typically used for live FM, such as fluorescent proteins, are likewise incompatible with conventional EM sample preparation techniques \cite{de2015correlated}. Although protocols have been developed for retaining fluorescence post-embedding \cite{kukulski2011correlated, watanabe2011protein, peddie2014correlative, fu2020meosem}, the same compromises exist between fluorescence retention and ultrastructure preservation. Fluorescent proteins have the additional limitation that the specimen must be genetically modified, rendering them unsuitable for use in native animals and humans. In-resin fluorescence preservation thus remains a challenge—only made more difficult by imposing high vacuum conditions, which may lower fluorescence intensities for biological probes typically optimized for use in aqueous environments \cite{peddie2014correlative}. We are nevertheless confident that future developments in fluorescent proteins and embedding media will present compelling opportunities to apply integrated array tomography to a variety of biological questions.

We foresee that the multimodal datasets obtained using this method will be instrumental in forthcoming machine learning applications \cite{eckstein2020microtubule, liu2020automatic, heinrich2021whole}. Thus far, applications of registered EM-FM datasets appear to be limited to facilitating registration of sequential CLEM data using artificial predictions for the fluorescence signal \cite{ounkomol2018label, seifert2020deepclem}. Volume EM datasets, particularly in connectomics, are now routinely segmented via deep convolutional neural networks \cite{buhmann2021automatic, heinrich2021whole}. Acquisition rates and manual annotation of datasets, however, both serve as bottlenecks for reconstructing dense networks of cells and organelles \cite{kornfeld2018progress}. Given its ability to provide labeled biological information as well as reduce imaging volumes to select regions, integrated array tomography is poised to deliver significant gains in this arena.

Future work will be directed towards further refinement and automation. The CL registration procedure could be made more robust by illuminating the sample with a greater number of CL spots or by increasing the camera integration time. Updates to the alignment software could furthermore allow for the distortion field correction used in \textcite{haring2017automated} to achieve sub-\SI{5}{\nano\meter} overlay precision. Cutting sections manually remains a significant bottleneck for throughput, as it is prone to error and requires expert training \cite{wanner2015challenges}. We expanded from a single section to nine, to 63, and have now placed more than 100 serial sections onto ITO-coated coverslips. Increasing beyond ${\sim}$\SI{10}{\micro\meter} of biological material, however, is cumbersome without more sophisticated sectioning techniques such as automated tape-collecting ultramicrotome (ATUM) \cite{hayworth2014imaging} or magnetic collection \cite{templier2019magc}. These may introduce their respective complications; ATUM, for example, is designed to collect sections on (opaque) Kapton tape. More extensive automation strategies can alternatively be applied to the correlative imaging pipeline. \textcite{delpiano2018automated} devised a way to automatically detect fluorescent cells using an integrated light and electron microscope. We envision a workflow for fully automated integrated array tomography in which fluorescent ROIs are automatically recognized, navigated to, and acquired, rendering three-dimensional CLEM datasets tailored to answer the specific biological research question.
