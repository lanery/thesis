\section{Introduction}
\label{sec:2.1_intro}

Mapping the full ultrastructural layout of complex biological systems at nanometer-scale resolution is a major challenge in cell biology. Electron microscopy (EM) is uniquely capable of stretching the vast spatial scales necessary to identify macromolecular complexes, subcellular structures, and intercellular architecture. As a consequence, interest in large-scale EM, where many high-resolution tiles are stitched into a gigapixel image frame, has exploded in recent years. Large-scale EM, however, suffers from the long acquisition times necessary to acquire sufficient signal at high resolution \cite{peddie2014exploring}.

A variety of approaches have been undertaken to advance throughput. While throughput is already a bottleneck for large-scale 2D imaging \cite{kuipers2015scanning}, most of these approaches have been developed under the framework of 3D imaging. Throughput is particularly relevant to the field of connectomics in which it typically takes months to acquire the image data necessary for neuronal reconstruction \cite{kornfeld2018progress}. To image the brain of a larval zebrafish, for example, \textcite{hildebrand2017whole} conducted multiple imaging rounds at successively higher magnification. Regions of interest (ROI) were selected between imaging rounds for successive, targeted acquisitions down to \SI{4}{\nano\meter\per\pixel} resolution, thereby reducing the time it would otherwise take to fully image the full brain at high resolution. Similarly, \textcite{delpiano2018automated} used detection of in-resin preserved fluorescence in an integrated light and electron microscope for automated guiding to ROIs for subsequent acquisition. Other approaches involve parallelizing the imaging load across multiple instruments. This has been employed in focused ion beam scanning electron microscopy (FIB-SEM) for the reconstruction of thick slices of Drosophila brain tissue at isotropic (\SI{8}{\nano\meter} $\times$ \SI{8}{\nano\meter} $\times$ \SI{8}{\nano\meter}) voxel resolution \cite{hayworth2015ultrastructurally} as well as in serial section transmission electron microscopy (ssTEM) for the yearlong acquisition of a cubic millimetre of mouse brain tissue \cite{yin2019petascale}. Dedicated instrumentation for faster imaging of serial thin sections has also been developed in recent years. In some instances conventional microscopes have been equipped with specialized detection optics to allow for larger fields of view \cite{bock2011network, zheng2018complete}. Multi-beam instruments in which a sample is simultaneously imaged by multiple focused electron beams have also been developed \cite{eberle2015high, ren2016transmission}.

Faster imaging could also be achieved by increasing signal collection in established thin sections approaches, which would allow for reduced acquisition time while maintaining a sufficient signal-to-noise ratio (SNR). It has previously been shown that the use of a retarding field increases SNR in SEM \cite{paden1968retarding, phifer2009improving}, but for biological imaging the use of a retarding field has thus far been investigated in detail only for serial blockface scanning EM (SBF-SEM) \cite{ohta2012beam, titze2013automated, bouwer2016deceleration}. Additionally, a high negative bias potential is employed in the Zeiss multibeam to allow for secondary electron (SE) detection from individual beamlets \cite{eberle2015high}. Conversely, the use of a positive stage bias has been examined for the suppression of secondary electrons \cite{xu2017enhanced}. The full benefits of stage bias remain underutilized because optimization criteria and signal detection in a magnetic immersion field, in particular, have yet to be addressed.

In the cases in which a negative bias potential has been used, a voltage is applied to the stage while the pole piece of the electron microscope is kept at ground such that an electric field is generated between the specimen and detector planes. While the primary electron beam experiences a deceleration, the signal electrons experience an acceleration from the specimen towards the dedicated detector. The ensuing acceleration results in an increase to the collected signal \cite{vsakic2011boron} and—if the detector geometry, landing energy, and potential bias are tuned properly—can be used to filter out secondary electrons \cite{bouwer2016deceleration}. The same signal can then be obtained with a shorter acquisition time.

Identification of biological structures and molecules in large-scale EM is typically complemented with approaches to label and visualize specific biomolecules or organelles. Aside from immuno-EM and genetically-encoded enzymatic tags that can deposit osmiophilic polymers, CLEM is perfectly suited to identify entities across spatial scales (reviewed in \textcite{de2015correlated}). However, if one wants to avoid intermediate processing of the sample, the sample preparation protocol must be adjusted to limit concentrations of heavy metal staining to prevent quenching of fluorophores \cite{kuipers2015scanning}. The reduced amount of staining material then needs to be countered by increased dwell time, further necessitating optimization of EM signal collection. Here we present faster imaging of tissue sections that have been prepared following conventional array tomography protocols through the use of a negative bias potential. No post-staining was applied as the tissue was immunostained for fluorescence post-sectioning.
