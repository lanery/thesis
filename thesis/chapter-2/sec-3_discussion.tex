\section{Discussion}
\label{sec:2.3_discussion}

We have shown that the SNR of a \SI{1}{\micro\second\per\pixel} image subject to a bias potential outperforms that of a \SI{5}{\micro\second\per\pixel} unbiased image or \SI{20}{\micro\second\per\pixel} in the case of immersion mode. This has important ramifications for large-scale and volume EM studies in which throughput is a primary concern. Due to practical limitations on time, it is often the case that large-scale EM studies are conducted on a single specimen. Negative potential bias facilitates comparison studies by allowing for multiple specimens to be acquired in the same timeframe that would otherwise be necessary for a single specimen. Experiments on specimens prone to electron beam irradiation damage are likewise facilitated as the same SNR can be achieved with a considerably smaller electron dose. Furthermore, a negative bias potential has recently been utilized to deliver enhanced EM contrast to tissue sections in which the fluorescence is preserved \cite{vos2021retarding}. Due to the minimal amounts of heavy metal staining \cite{kukulski2011correlated}, such samples have thus far been challenging to image---in certain instances requiring dwell times of up to \SI{60}{\micro\second\per\pixel} \cite{peddie2014correlative}.

Our simulations show that BSE collection is enhanced by an effective increase of the detector numerical aperture---by applying the bias potential we increase the range of angular distributions of the BSEs able to be collected. However, this does not fully explain the extent of the increase in SNR observed experimentally. In particular, the simulations predict roughly a factor two increase in signal collection as the bias voltage is raised to our maximum of \SI{3}{\kilo\volt}, while our empirical measurements show SNR improvements of one to two orders of magnitude. This disparity can be explained in part by the electron gain factor of the detector. \textcite{vsakic2011boron} shows that the signal generated in the detector by the incident electrons increases linearly with energy between \SIrange{200}{10000}{\electronvolt}. Thus, in addition to increasing the amount of collected BSEs, the bias potential also leads to signal enhancement via BSE acceleration. At low bias voltages the images appear to be dominated by one particular source of noise---which we suspect derives from the scanning electronics. Increasing the bias potential in this regime leads to an exponential rise in the SNR as this noise source is drowned out (Figure \ref{fig:2.4_snr}; Figure \ref{fig:2.5_noise}). At sufficiently high bias voltages, the image noise is instead dominated by shot noise, constraining the exponential rise in SNR beyond \SI{1.5}{\kilo\volt}. The practical limit to the amount of bias potential we are able to apply is limited by the dielectric breakdown in vacuum. We estimate for our particular setup that the breakdown voltage occurs above \SI{3}{\kilo\volt}---well beyond the point at which the SNR plateaus.

Other volume EM methods such as SBF-SEM or FIB-SEM also stand to gain from the use of a negative potential bias. The gains in imaging speeds have the potential to shift the bottlenecks in these approaches to overhead factors such as time spent slicing or milling \cite{kornfeld2018progress}. If unaccounted for, the non-planar geometries in these techniques may induce more pronounced charging artefacts. \textcite{bouwer2016deceleration} show that charging artefacts can be mitigated by successfully filtering out SEs. We have found that SE filtering is accomplished at moderate potential biases, though in \textcite{bouwer2016deceleration} the innermost rings of the BSE detector had to be selectively turned off to achieve the same effect. Negative bias potential could similarly be combined with the multi-scale approach taken by \textcite{hildebrand2017whole}. The combination with an integrated microscope as demonstrated here could then offer a further benefit by in-situ selection of the regions of interest for high magnification acquisition. We envision a strategy in which regions of interest are first identified via fluorescence microscopy, then automatically navigated to and imaged with high resolution EM \cite{koning2019integrated}. Higher throughput could then be realized through a combination of faster acquisition via the negative bias potential, the removal of additional rounds of imaging, and the elimination of overhead from the entire imaging pipeline.

Further throughput enhancement could be obtained in several ways. One option would be to increase the beam current, thus increasing the per-pixel electron dose. Higher currents, however, require larger aperture sizes which result in greater chromatic and spherical aberration. This can be problematic for many biological applications in which keeping aberrations at a minimum is critical for reaching a desired resolution, e.g. resolving neuronal connections, nuclear pores, or cell–cell junctions. Hence, it only makes to image with the maximum current acceptable for one’s application. At the same time, the use of a negative bias potential has previously been shown to result in improved resolution due to reduced space charge and aberrations \cite{paden1968retarding, mullerova2003scanning}. Thus, the use of a negative bias potential may allow for a marginally higher beam current to further increase throughput. Alternatively, the signal may be strengthened by increasing the landing energy. This may also be disadvantageous---as evidenced in Figure \ref{fig:2.1_setup} and Figure \ref{fig:2.6_cells}---since too great a landing energy will result in partial transmission of electrons through the tissue section. In addition to reducing the number of generated BSEs in the tissue, this will increase the noise level by detection of accelerated BSEs generated in the underlying substrate. Finally, more signal could be generated by increasing the amount of staining material in the sample. This is a common approach for certain applications within large-scale EM such as neuronal connectomics, where an almost binary level of contrast may still be acceptable \cite{kuipers2015scanning}. Our stage bias approach holds promise to decrease acquisition times also in these applications, provided the lower limit imposed on dwell time by the detector response time is not reached.
