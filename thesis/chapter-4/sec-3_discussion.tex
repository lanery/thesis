\section{Discussion}
\label{sec:4.3_discussion}

We have demonstrated the ability of a CNN to artificially predict biological labels in electron microscopy images based on registered CLEM training data. This has important ramifications for many areas within cell biology in which additional labelling techniques are implemented to facilitate recognition of structures in EM \cite{de2015correlated}. In order to generate label predictions, registered EM-FM image pairs are required to train the CNN. Although in this work the accumulation of correlative datasets was facilitated by integrated CLEM, this is not a pre-requisite. Sequential CLEM methods in which light and electron microscopy are performed by different instruments in succession may also be suitable. It is unknown, however, what effects might come about from the use of fiducial markers and potentially less precise image registration across large fields of view.

While label predictions are not generalizable to arbitrary organelles outside of the training dataset, we have shown that the network is capable of transfer learning across cell types. Predictions on mouse breast tumor cell nuclei were made after supplementing a training dataset comprised primarily of rat pancreas tissue with a limited amount of correlative data from tumor cells. Aided by data augmentation, label predictions were furthermore found to be robust to changes in EM imaging parameters, additional shot noise, and sectioning artefacts. By further supplementing existing correlative datasets with data from different organisms, cell types, and microscopes, robustness could be improved even further.

The measured fluorescence and CLEMnet predicted labels fall short of providing adequate templates for fully automated segmentation. Nevertheless, we have shown that fluorescence labels are not only capable of facilitating annotation, but that as part of an image processing pipeline, they enable a framework for semi-automatic, weakly supervised segmentation. It is difficult to imagine that a deep CNN trained on automatically generated segmentation masks (i.e. no manual annotation whatsoever) could outperform the same network when trained on manually generated segmentation masks in the near future \cite{dorkenwald2017automated, januszewski2018high, roels2019domain}. Even so, semi-automated and fully automated approaches may still fulfill a role in segmenting biological image data. For smaller-scale applications in which training datasets are still tractable, a segmentation model based on manually segmented organelles is likely the more sensible approach. But for large-scale or volume applications in which a pixel-perfect segmentation may not be strictly necessary, a semi-automated labelling approach may offer valuable time-savings at the cost of precision.

The deep CNN developed here offers a means to automate fluorescence-like labelling of electron microscopy data at negligible cost with respect to time, effort, and money. Once the network has been sufficiently trained, label predictions can be automatically generated in seconds. This allows research facilities to process only a handful of sections for correlative fluorescence and electron microscopy, while preparing the rest of the sample for EM only. The entire EM volume could then be overlaid with fluorescence-like labels after training on the portion of the volume set aside for correlative imaging. In addition, EM datasets could be labelled with a larger number of distinct labels than would be allowed in a single fluorescence experiment by simply labelling different targets in different subsets of the sample.  Alternatively, it would enable comparative studies of multiple samples imaged by EM (e.g. \cite{sokol2015large, de2020large}) to be given biological labels virtually for free, providing biological insight to a wealth of grayscale data. 
