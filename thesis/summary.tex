\chapter*{Summary}
\addcontentsline{toc}{chapter}{Summary}
\setheader{Summary}

% --- Blind text ---
% Summary in English. \blindtext

% A new paragraph with a citation \cite{hildebrand2017whole}. Let us also add some numbers with units like \SI{100}{\milli\gram} and even a range of units \SIrange{14}{27}{\kilo\pascal}. \blindtext Could also look at units in math mode like $\SI{13.8}{\newton\per\metre}$. Let's follow this up with some math \cite{delpiano2018automated}.

% \blindmathpaper




% --- Outline ---
% Importance of multi-modal imaging
% What is array tomography
% Limitations of array tomography
% Does, however, come with challenges --> tougher sample prep
% Have to compromise on staining --> less contrast --> longer integration times
% So before diving into a workflow we first have to figure out a way to image such that times are reasonable
% Negative potential bias
% Also filters out SEs
% Can show that we can image 20x faster while maintaining SNR
% So we can incorporate this speed up into our 3D CLEM workflow

% --- Actual summary ---
% [This thing] [puts forth / establishes] an integrated microscopy workflow for the three-dimensional reconstruction of biological material from correlative light and electron microscopy (CLEM) image data.


% Why is this relevant? 
Multi-modal imaging techniques have become essential for better understanding fundamental questions in cell biology such as %[good question] or 
disease progression. While individual microscopy methods have rapidly advanced in recent years, the information content of any one imaging technique is limited to the type of contrast that particular technique is sensitive to. By tagging particular biomolecules with a fluorescent protein, fluorescence microscopy (FM), for example, can relay dynamic information about the distribution of these biomolecules in their cellular environment. It struggles, however, to convey information regarding the structure of the organelles that might contain these biomolecules or the surroundings of their cellular environment. Electron microscopy (EM), on the other hand, can provide detailed layouts of cellular structure by staining membranes with heavy metals. Thus, by correlating these modalities (correlative light and electron microscopy, CLEM), a more holistic understanding of the relationship between structure and function at the (sub-)cellular level can be achieved. 
% In this thesis, we present a correlative imaging method for rendering three-dimensional CLEM volumes of ___ using an integrated light and electorn microscope.

% [scaling the methods from single cells to swaths of tissue or whole organs] 

% we can learn a lot more intercellular stuff].

Array tomography (AT) is a technique combining FM and EM for volumetric imaging, first introduced in 2007 for studying brain tissue. The technique has since expanded% to target everything from [something] to [something]
, but the approach has largely remained the same. Biological material is cut into a series of ultrathin (${\sim}$\SI{100}{\nano\meter}) sections (an array) and prepared for sequential FM and EM imaging by applying a series of immunofluorescence and heavy metal stains. Correlative images of the serial sections are then computationally aligned to reconstruct the 3D structure (tomography). Compared to other volumetric imaging techniques in the life sciences, AT offers the ability to correlate structure and function at high resolution across large fields of view. Moreover, it enables high axial resolution% with the added advantage that the [axial resolution is matched with EM as it is] 
for both EM and FM as determined by the section thickness.

While AT is an incredibly useful technique for% [volume CLEM imaging / finding rare events in brain tissue]
volume CLEM imaging, it poses several challenges. Finding back regions of interest (ROI) across imaging modalities is nontrivial, particularly when the ROI might be several microns in size, scattered about millimeters of tissue. Moreover, correlating the datasets is complex and must be done manually. Finally, intermediate sample preparation between FM and EM is both tedious and prone to cause specimen shrinkage, complicating the already difficult correlation. One means of combating these obstacles is by merging the separate imaging systems into a single, integrated microscope. In 2013, a fluorescence microscope was thus retro-fitted into the vacuum chamber of a scanning electron microscope (SEM) at TU Delft. This enabled quasi-simultaneous FM and EM imaging of the same field of view as well as synchronized stage movements when navigating about the sample. Several years later, an automated registration procedure was developed for automatically and precisely overlaying the fluorescence signal onto the EM.% These advantages and opportunities for automation are made possible by integrating.

Integration, however, is not without its own limitations. Chief among them are the constraints imposed on sample preparation. Typical EM sample processing involves staining a biological specimen with heavy metals to provide a contrast mechanism and embedding it in a polymer resin such that it can be cut into thin sections. However, the heavy metals and cross-linking of polymers necessary for EM have the unfortunate consequence of quenching the fluorophores needed for FM. In order to retain fluorescence, protocols must therefore be adapted to alternate%, less-[something] 
resins or to limit the concentration of heavy metals, which inherently results in decreased signal generation. While signal loss can be compensated for by increasing the acquisition time, the acquisition times necessary to maintain an adequate signal-to-noise ratio (SNR) for analysis would be untenable. Prior work involving EM imaging of serial sections in which the fluorescence is preserved required frame times of several minutes. To image even something as small as a single mammalian cell (say ${\sim}\text{20} \times \text{20} \times \text{20}$ \si{\micro\meter^3}) would then require ${\sim}$\SI{2}{days} assuming \SI{5}{\nano\meter} lateral, \SI{100}{\nano\meter} axial resolution. As the biological systems we would like to study involve hundreds if (not thousands) of cells, the time scales begin to exceed the duration of a typical PhD project in the Netherlands.

Prior to engineering a workflow for imaging volumes of tissue with an integrated microscope, it was therefore necessary to first improve the data acquisition rate. The use of a negative bias potential applied to the specimen stage had previously been shown to enhance signal collection, allowing for reduced acquisition times and thus faster imaging speeds. Previous applications of specimen bias, however, were largely limited to tuning the electron penetration depth in block-face imaging, or non-biological applications. We thus optimized the use of a negative potential bias for serial section EM. The bias potential works by accelerating backscattered electrons (BSEs) to a dedicated detector, enhancing the generated signal. We showed via charged particle optics modeling that this has the simultaneous effect of filtering out secondary electrons (SEs). This is advantageous as SEs carry topographic information regarding the surface of the specimen, while BSEs scatter on the heavy metals bound to the cell membranes, revealing detailed structural information. By applying the optimized bias, we were able to achieve the same SNR with a 20-fold decrease in dwell time. Acquisitions that might have previously taken weeks, could then be completed in a matter of hours.

Thus, we can begin to engineer an array tomography workflow for acquiring volume CLEM data using an integrated microscope. A proof of concept was carried out on pancreas tissue from rat and zebrafish specimens. Serial sections were cut and prepared for simultaneous FM and EM imaging, and the insulin granules within the endocrine region of the pancreas (islets of Langerhans) were immunolabeled with a fluorescent dye. To facilitate navigation between serial sections in the integrated microscope, an image processing pipeline for segmenting serial sections was established. Automated imaging routines incorporating the EM-FM registration procedure were then developed to acquire correlative datasets with high overlay precision. The fluorescence of the insulin granules allowed for straightforward identification of our chosen ROI, the islet of Langerhans. By limiting the acquisition of high-resolution EM images to regions expressed by fluorescence, we were able to further expedite total acquisition times. Correlative alignment routines were developed to reconstruct portions of the islet in 3D.

The workflow for integrated AT enabled the acquisition and reconstruction of large-scale correlative datasets, assisting recognition of organelles and certain sub-cellular features within select ROI. As this facilitated interpretation of EM data, we questioned whether we could leverage recent advances in deep learning to supplement (unlabelled) EM data with biological labels. We therefore used our correlative datasets to train a convolutional neural network (CNN) to generate artificial fluorescence predictions. It was found that the predictions generated by the CNN were highly correlated with the measured fluorescence. And because the artificial fluorescence signal generated by the network is localized to specific organelles, we were able to devise strategies for segmenting organelles within these datasets without the need for extensive manual intervention.

% Challenges that still lie ahead.

% Paired with the high-accuracy fluorescence-to-EM registration that can be obtained consistently over large areas, integrated microscopes seem particularly suited to improve throughput and functional mapping in serial sections volume-EM. Instrumentation seems to be in place, but automation, especially in fluorescence recognition and unattended acquisition, needs development. Challenges also remain in further, more wide-spread applications of fluorescence preserving EM sample preparation, on-section immuno-labelling, and reduction of resin auto-fluorescence.



\chapter*{Samenvatting}
\addcontentsline{toc}{chapter}{Samenvatting}
\setheader{Samenvatting}

Multimodale beeldvormingstechnieken zijn essentieel geworden voor een beter begrip van fundamentele vragen in de celbiologie, zoals ziekteprogressie. Hoewel individuele microscopiemethoden de afgelopen jaren snel zijn gevorderd, is de informatie-inhoud van elke beeldvormingstechniek beperkt tot het type contrast waarvoor die bepaalde techniek gevoelig is. Door bijvoorbeeld bepaalde bio-mole-culen te labelen met een fluorescerend eiwit kan fluorescentiemicroscopie (FM) dynamische informatie geven over de verdeling van deze biomoleculen in hun cellulaire omgeving. Het heeft echter moeite om informatie over te brengen over de structuur van de organellen die deze biomoleculen of de omgeving van hun cellulaire omgeving kunnen bevatten. Elektronenmicroscopie (EM), aan de andere kant, kan gedetailleerde indelingen van de celstructuur opleveren door membranen te kleuren met zware metalen. Door deze modaliteiten te correleren (correlatieve licht- en elektronenmicroscopie, CLEM), kan een meer holistisch begrip van de relatie tussen structuur en functie op (sub)cellulair niveau worden bereikt.

Array tomografie (AT) is een techniek die FM en EM combineert voor volume-trische beeldvorming, voor het eerst geïntroduceerd in 2007 voor het bestuderen van hersen-weefsel. De techniek is sindsdien uitgebreid, maar de aanpak is grotendeels hetzelfde gebleven. Biologisch materiaal wordt in een reeks ultradunne (${\sim}$\SI{100}{\nano\meter}) secties (een array) gesneden en voorbereid voor sequentiële FM- en EM-opnames door een reeks immunofluorescentie en zware metaal kleuringen. Correlatieve beelden van de seriële secties worden vervolgens computationeel uitgelijnd om de 3D-structuur (tomografie) te reconstrueren. Vergeleken met andere volumetrische beeldvormingstechnieken in de biowetenschappen, biedt AT de mogelijkheid om structuur en functie met hoge resolutie over grote gezichtsvelden te correleren. Bovendien maakt het een hoge axiale resolutie mogelijk voor zowel EM als FM, zoals bepaald door de sectiedikte.

Hoewel AT een ongelooflijk nuttige techniek is voor volume-CLEM-beeld-vorming, brengt het verschillende uitdagingen met zich mee. Het terugvinden van interessegebieden (ROIs) in de verschillende beeldvormende modaliteiten is niet triviaal, vooral wanneer de ROIs enkele microns groot kunnen zijn, verspreid over millimeters weefsel. Bovendien is het correleren van de datasets complex en moet het handmatig gebeuren. Ten slotte is de tussentijdse preparaatvoorbereiding tussen FM en EM zowel tijdrovend als vatbaar voor krimp van het preparaat, wat de toch al lastige correlatie verder bemoeilijkt. Een manier om deze obstakels te overwinnen is door de afzonderlijke microscopie technieken samen te voegen tot één geïntegreerde microscoop. Zo is in 2013 aan de TU Delft een fluorescentiemicroscoop ingebouwd in de vacuümkamer van een scanning elektronenmicroscoop (SEM). Dit maakte quasi-simultane FM- en EM-beeldvorming van hetzelfde gezichtsveld mogelijk, evenals gesynchroniseerde microscoop-tafel bewegingen bij het navigeren in het preparaat. Enkele jaren later werd een geautomatiseerde registratieprocedure ontwikkeld om het fluorescentiesignaal automatisch en nauwkeurig over de EM te leggen.

Integratie is echter niet zonder beperkingen. De belangrijkste daarvan zijn de beperkingen die worden opgelegd aan de preparaatvoorbereiding. Typische verwerking van EM-preparaten omvat het kleuren van een biologisch preparaat met zware metalen om voor een contrastmechanisme te zorgen en het in te bedden in een polymeerhars zodat het in dunne secties kan worden gesneden. De zware metalen en verknoping van polymeren die nodig zijn voor EM hebben echter het ongelukkige gevolg dat de fluoroforen die nodig zijn voor FM hun werking verliezen. Om fluorescentie te behouden, moeten daarom protocollen worden aangepast om harsen af te wisselen of om de concentratie van zware metalen te beperken, wat inherent resulteert in een verminderde contrast (in de EM?). Hoewel contrastverlies kan worden gecompenseerd door de opnametijd te vergroten, zouden de opnametijden die nodig zijn om een adequate signaal-ruisverhouding (SNR) voor analyse te behouden, onwerkbaar zijn. Eerder werk met EM-opnames van seriële secties waarin de fluorescentie behouden blijft, vereiste en opnametijd per beeld van enkele minuten. Voor het in beeld brengen van zoiets kleins als een enkele zoogdiercel (bijvoorbeeld ${\sim}\text{20} \times \text{20} \times \text{20}$ \si{\micro\meter^3}), zou dan ${\sim}$\SI{2}{dagen} nodig zijn, uitgaande van een \SI{5}{\nano\meter} laterale en \SI{100}{\nano\meter} axiale resolutie. Omdat de biologische systemen die we willen bestuderen honderden, zo niet duizenden cellen omvatten, beginnen de tijdschalen de duur van een gemiddeld Nederlands promotieonderzoek te overschrijden.

Het was daarom noodzakelijk om eerst de opname-snelheid te verbeteren, voordat we een workflow ontwikkelen voor het afbeelden van volumes weefsel met een geïntegreerde microscoop. Het is aangetoond dat het gebruik van een negatieve voorspanning op de microscoop-tafel het contrast in de EM beelden verbetert, waardoor de opnametijden worden verlaagd en dus snellere opnamesnelheden mogelijk zijn. Eerdere toepassingen van een voorspanning op de microscoop-tafel waren echter grotendeels beperkt tot het afstemmen van de elektronen\-penetratie\-diepte in de beeldvorming van niet-biologische samples, of bij seriële Block-Face microscopie, waarbij afwisselend het oppervlak van het preparaat wordt afgebeeld en een plakje van het preparaat wordt weggesneden. Daarom hebben wij dus het gebruik van een negatieve voorspanning voor seriële sectie EM geoptimaliseerd. De voorspanning werkt door terugverstrooide elektronen (BSE's) te versnellen naar een speciale detector, waardoor het gegenereerde signaal wordt versterkt. We laten door middel van modellering van geladen deeltjesoptica zien dat dit het gelijkertijd  ook de secundaire elektronen (SE's) eruit filtert. Dit is voordelig omdat SE's topografische informatie over het oppervlak van het preparaat bevat, terwijl BSE's verstrooid worden door de zware metalen die aan de celmembranen zijn gebonden, waardoor gedetailleerde structurele informatie wordt onthuld. Door de geoptimaliseerde voorspanning toe te passen, konden we dezelfde SNR bereiken met een 20-voudige afname van de opnametijd. Acquisities die voorheen weken in beslag namen, konden vervolgens binnen enkele uren worden afgerond.

Vervolgens kunnen we dus beginnen met het ontwerpen van een array-tomografie workflow voor het verkrijgen van volume-CLEM data met behulp van een geïntegreerde microscoop. Een proof of concept werd uitgevoerd op preparaten van pancreasweefsel uit ratten en zebravissen. Er werden seriële secties gesneden en voorbereid voor gelijktijdige FM- en EM-microscopie, en de insulinekorrels in het endocriene gebied van de pancreas (eilandjes van Langerhans) werden immunogelabeld met een fluorescerende kleurstof. Om de navigatie tussen seriële secties in de geïntegreerde microscoop te vergemakkelijken, werd een beeldverwerkingsprocedure opgezet voor het segmenteren van seriële secties. Geautomatiseerde opname-routines waarin EM-FM-registratieprocedures zijn opgenomen, werden vervolgens ontwikkeld om correlatieve datasets met een hoge overlay-precisie te verkrijgen. De fluorescentie van de insulinekorrels zorgde voor een eenvoudige identificatie van onze gekozen ROI, het eilandje van Langerhans. Door de acquisitie van hoge-resolutie EM-beelden te beperken tot regio's waarin zich fluorescentie bevindt, waren we in staat om de totale opnametijd verder te verkleinen. Er werden correlatieve uitlijningsroutines ontwikkeld om delen van het eilandje in 3D te reconstrueren.

De workflow voor geïntegreerde AT maakte de acquisitie en reconstructie van grootschalige correlatieve datasets mogelijk, wat helpt bij de herkenning van organellen en bepaalde subcellulaire functies binnen geselecteerde ROI. Omdat dit de interpretatie van EM-data vergemakkelijkte, vroegen we ons af of we recente ontwikkelingen in Deep Learning konden gebruiken om (niet-gelabelde) EM-gegevens aan te vullen met biologische labels. We hebben daarvoor onze correlatieve datasets gebruikt om een convolutioneel neuraal netwerk (CNN) te trainen om kunstmatige fluorescentievoorspellingen te genereren. Het bleek dat de voorspellingen die door de CNN werden gegenereerd sterk gecorreleerd waren met de gemeten fluorescentie. En omdat het kunstmatige fluorescentiesignaal, dat door het netwerk wordt gegenereerd, is gelokaliseerd op specifieke organellen, waren we in staat om strategieën te bedenken voor het segmenteren van organellen binnen deze datasets zonder uitgebreide handmatige tussenkomst.
