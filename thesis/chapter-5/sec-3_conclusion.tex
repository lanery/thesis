\section{Conclusion}
\label{sec:5.3_conclusion}

The goal of this thesis was to develop a method for 3D CLEM utilizing an integrated light and electron microscope. We began with the realization that conventional SEM imaging of weakly-stained specimen prepared for fluorescence microscopy required highly impractical dwell times. Thus, we implemented a negative bias potential to enhance the backscattered electron (BSE) yield. An empirical optimization of the stage bias, informed by electron optics simulations, ultimately led to orders of magnitude improvement in the signal-to-noise (SNR) ratio. This solved the throughput problem for large-scale, integrated CLEM, which allowed us to develop a scalable workflow for integrated array tomography. The method we created further expedited acquisitions by limiting high-resolution EM to select ROI targeted by fluorescence expression, while high precision EM-FM overlay is achieved using cathodoluminescent markers \cite{haring2017automated}. 

After successfully demonstrating our workflow on pancreatic tissue from different organisms, we explored the potential for large-scale CLEM datasets to be used as training data for machine learning applications. It was found that a modified U-net \cite{ronneberger2015u}, trained on such correlative datasets, was capable of generating high-fidelity predictions of the fluorescence signal from EM. As linking structure to function is one of the primary goals in cell biology, we then tested whether this data would be advantageous in organelle segmentation. While naive approaches of thresholding the measured and predicted fluorescence signals for use as segmentation masks proved unremarkable, more complicated mask generation techniques seem promising. Finally, we discussed the ways in which integrated array tomography could be expanded into a fully automated workflow as well as potential applications for multibeam SEM. Coupling these improvements together with advances in fluorescent probes would bring integrated array tomography to the forefront of volume CLEM, lending new insight to a host of questions within structural biology.
